%% RESUME - Andrew K Davis
\documentclass[10pt,final,sans]{resume}
\usepackage{outlines}
\usepackage{tabularx}
\usepackage{changepage}   % for the adjustwidth environment

\begin{document}
\setlength\headheight{28pt} % make header tall enough
\name{ANDREW K. DAVIS}
\lcontact{
  \begin{tabular}{@{}ll@{}}
    \faLinkedin  & \href{https://www.linkedin.com/in/andrew-davis-1a293b137/}{andrewdavis} \\
    % \faUser      & \href{https://andrewkdavis.com}{andrewkdavis.com} \\
    \faGithub    & \href{https://github.com/andrewkdavis97}{andrewkdavis97} \\
  \end{tabular}
}
\rcontact{
  \begin{tabular}{@{}r@{}}
    andrewkdavis@protonmail.com \\
    115 Solar Street, Apartment 407 \\
    Syracuse, NY 13204 \\
    (315) 630-8565 \\
  \end{tabular}
}
\center{
  US Citizen with Security Clearance\\
}

\section{Professional Summary}
\begin{adjustwidth}{.3cm}{}
  \raggedright{I am an FPGA Engineer with 5 years of VHDL industry experience. I am looking for opportunities to accelerate my technical proficiency and growth. I have been both an individual contributor since June 2019, and I have been a team lead since October 2023. My work has been primarily with Zynq Ultrascale+ family of devices.}
\end{adjustwidth}

\vfill

\section{Education}
\headerwithlabel{ Clarkson University } { Potsdam, NY } { August 2015 -- May 2019 }
\begin{adjustwidth}{.3cm}{}
  GPA 3.520, Bachelor of Science in Computer Engineering\\
  \headerwithlabel{ Undergraduate Research } { 3D Fingerprint Scanning and Modeling } { }
  Created a prototype to photograph fingers and output 3D renders. Helped develop an algorithm in C++ to unwrap the 3D renders into nail-to-nail fingerprint rolls. Presented at the 2017 IARPA N2N Challenge in Laurel, Maryland.
\end{adjustwidth}

\vfill

\section{Work Experience}
\headerwithlabel{ SRC Inc } { Syracuse, NY } { June 2019 -- Present }
\begin{adjustwidth}{.3cm}{}
\begin{itemize}
  \item[$\bullet$] As team lead: oversee a team of 2-3 engineers; lead daily standups; assign bi-weekly tasking. 
  \item[$\bullet$] As individual contributor: architecture and documentation; coding and simulation; hardware integration. 
  \item[$\bullet$] As interview team-member: performed over 20 interviews, where I evaluate technical skill; and evaluate culture fit. 
\end{itemize}
\end{adjustwidth}

\begin{adjustwidth}{.3cm}{}
  \headerwithlabel{ RF Front End } { MPSoC, XCAU15P, Team Lead } { October 2023 -- Present }
  \begin{itemize}
    \item[$\bullet$] Lead a team of 3 engineers to design, simulate, and implement firmware targetting the FPGA.
    \item[$\bullet$] FPGA design for Artix Ultrascale+ RF Front End card. Matched hardware requirements to FPGA IO planning and clock management.
    \item[$\bullet$] Designed firmware that queues RF Control MORA messages and communicates with off-card hardware (filter shift registers, amplifiers, attentuators). 
    \item[$\bullet$] Implemented manual clock routing for external clock source and performed post route timing analysis.
  \end{itemize}
\end{adjustwidth}

\begin{adjustwidth}{.3cm}{}
  \headerwithlabel{ SDR } { RFSoC, XCZU43DR, Lead } { May 2023 -- September 2023 }
  \begin{itemize}
    \item[$\bullet$] FPGA design for Zynq Ultrascale+ AMS WB3XR2 Dual RFSoC. Matched hardware requirements to FPGA IO planning. 
    \item[$\bullet$] Designed high level AXI4L memory map between RFSOC0 and RFSoC1. Documented the allocation of 4.25GB of user space addressable by each RFSoCs programmable logic.
    \item[$\bullet$] Performed initial bringup of card using TCL build scripts and Petalinux image using third party IP and Zynq US+ block design.
  \end{itemize}
\end{adjustwidth}

\begin{adjustwidth}{.3cm}{}
  \headerwithlabel{ DAC Stimulator } { RFSoC, ZCU111, Individual Contributor } { December 2020 -- May 2022 }
  \begin{itemize}
  \item[$\bullet$] Integrated 10GbE with SFP PHY/MAC. An external PDW generator sent ethernet packets containing transmit-scheduled pulse descriptor words. Designed message parsing logic on top of the extracted UDP data to extract the fields needed for transmit on the ZCU111 DACS.
  \item[$\bullet$] Designed AXI4-Lite controlled I/Q scaling. Scale factors were multiplied onto I/Q from full scale (1) down to zero. Polar inputs converted to complex I/Q and fed into complex multipliers against a single I/Q data stream from the transmit scheduler.
\end{itemize}
\end{adjustwidth}

\begin{adjustwidth}{.3cm}{} 
  \headerwithlabel{ Motor Control } { MPSoC, ZCU106, Individual Contributor } { December 2019 -- November 2020 }
  \begin{itemize}
  \item[$\bullet$] Motor control with DAC8771 (single channel, 16bit DAC). AXI4-Lite to SPI interface. Wrote to DAC registers via SPI from Zynq US+ PS interfaced through Python. Verification with Oscilloscope.
  \end{itemize}
\end{adjustwidth}

\begin{adjustwidth}{.3cm}{}
  \headerwithlabel{ SVD to LaTeX } {Python, Individual Contributor } { June 2019 -- November 2019 }
  \begin{itemize}
  \item[$\bullet$] Created SVD to LaTeX tool in Python, based off of CMSIS SVD specification. This tool generates memory-mapped documentation for AXI4-Lite regions: register, FIFO, and RAM memory styles.
  \end{itemize}
\end{adjustwidth}

\vfill

\section{Skills}
\begin{itemize}
  \item[$\bullet$] \textbf{Hardware}:  Zynq US+ RFSoC/MPSoC, Zynq7000, Artix US+ \\
  \item[$\bullet$] \textbf{Software}: VSCode, Git, JIRA, Perforce, Confluence, MS Visio, {\textrm \LaTeX}, Linux, Windows \\
  \item[$\bullet$] \textbf{Languages}: VHDL, TCL, Python (VUnit/CocoTB simulation, and integration scripting) \\
  \item[$\bullet$] \textbf{Functional}: Mentor, Strong Communicator, Team Player, Self Motivator, Book Reader \\
  % \textbf{Standards}  & AXI4-Stream, AXI4-Lite, Ethernet, SPI  \\
  % ZUS+ RFSoC (XCZU28DR:ZCU111?S, XCZU43DR, XCZU47DR), ZUS+ MPSoC (XCZU9EG:ZCU101, XCZU7EV:ZCU106), Zynq7000 (), Artix US+ ()
\end{itemize}

\vfill

\end{document}
