%% RESUME - Andrew K Davis
\documentclass[10pt,final,sans]{resume}
\usepackage{outlines}


\begin{document}
\setlength\headheight{28pt} % make header tall enough
\name{ANDREW K. DAVIS}
\lcontact{
  \begin{tabular}{@{}ll@{}}
    \faLinkedin  & \href{https://www.linkedin.com/in/andrew-davis-1a293b137/}{andrewdavis} \\
    \faUser      & \href{https://andrewkdavis.com}{andrewkdavis.com} \\
    \faGithub    & \href{https://github.com/andrewkdavis97}{andrewkdavis97} \\
  \end{tabular}
}
\rcontact{
  \begin{tabular}{@{}r@{}}
    andrewkdavis@protonmail.com \\
    115 Solar Street, Apartment 407 \\
    Syracuse, NY 13204 \\
    (315) 630-8565 \\
  \end{tabular}
}
\center{
  US Citizen with Security Clearance\\
}

\section{Professional Summary}
\raggedright{I am an FPGA Engineer with 5 years of VHDL industry experience. I am looking for opportunities to accelerate my technical proficiency and career growth. I have been both an individual contributor since 2019, and I have led a small team of firmware engineers since the beginning of 2024. My work has been primarily with Zynq Ultrascale+ family of devices, with a focus on DAC/ADC RF Data Converter and 10G Ethernet IP.}

\section{Skills}
\begin{itemize}
  \item Languages: VHDL, TCL, Python (for Vunit/CocoTB Simulation, and Hardware Integration)
  \item Standards: AXI4-Stream, AXI4-Lite, Ethernet, SPI
  \item Software: {\textrm \LaTeX}, Confluence, Git, Perforce, MS Visio, Linux
  \item Hardware: Zynq US+ RFSoC (XCZU28DR:ZCU111, XCZU43DR, XCZU47DR), Zynq US+ MPSoC (XCZU9EG:ZCU101, XCZU7EV:ZCU106), Zynq7000 (), Artix US+ ()
  \item Functional: Mentorship, Overview Tasking, JIRA Sprints, Daily Standups
\end{itemize}

\section{Education}
\headerwithlabel{Clarkson University}{Potsdam, NY}{May 2019}
Bachelor of Science in Computer Engineering | 3.520 GPA
\begin{tabular}{p{0.95\linewidth}}
  \vspace*{-3mm}
  \headerwithlabel{Undergraduate Research:}{3D Fingerprint Scanning and Modeling}{}
\raggedright{Created a prototype to scan and produce 3D renders of users fingerprints. Helped to develop an algorithm in C++ to project and unwrap the 3D renders into fingerprint rolls. Presented the prototype and research at the IARPA N2N Challenge 2017 in Laurel, Maryland.}
\end{tabular}

\section{Work Experience}
\headerwithlabel{SRC Inc}{Syracuse, NY}{October 2023 -- Present}
\textbf{Lead Firmware Engineer}
\begin{itemize}
  \item Interview team member for the Firmware Engineering department; performed over 20 interviews, evaluating technical skill and a candidate's overall fit for the company and department. 
  \item Act in a lead role to oversee a team of 2-3 engineers; lead daily standups and assign bi-weekly Sprint tasking. 
  \item Act as the responsible individual for Functional Overview architecture and documentation.
  \item Oversee Detailed Design architecture and documentation, coding and simulation, and hardware integration. 
  \item 
\end{itemize} 


\headerwithlabel{SRC Inc}{Syracuse, NY}{June 2019 -- Present}
\textbf{Firmware Engineer}\\
  NGEW 
  \begin{itemize}
    \item Created SVD to Latex Python command line tool to generate Latex. Based off the CMSIS SVD specification. Used to generate memory-mapped documentation for AXI4-Lite register, FIFO, and RAM memory style regions.
    \item Design/Code/Sim CORDIC Rotation that could accept pipelined requests. (what FPGA?)
  \end{itemize}

  RECU
  \begin{itemize}
    \item Design/Code/Sim DAC8771 (single channel, 16bit) AXI4-Lite to SPI interface. Also used Python script to configure and set output voltages via SPI, through AXI4-Lite commands issues through Zynq US+ PS. Verification performed with Oscilloscope. (what FPGA?)
    \item Design/Code/Sim for an electrical Resolver (azimuth/elevation/speed) feedback loop. (what FPGA?)
    \item Design/Code/Sim for an I/O Recoder IP. Could detect edge detections on GPIO and would record and store time deltas between transitions. Stored in BRAM and could be later played back for debugging efforts. (what FPGA?)
  \end{itemize}

  RFSTIM
  \begin{itemize}
  \item Implemented Xilinx RFDC to use 7 DACs across 2 Tiles. Integrated with IFGEN (custom transmit scheduler. took in frequency, pulse width, starttime, end time, and I/Q adjustments to generate static PDWs)
    \begin{itemize}
      \item Implemented I/Q scaling AXI4-Lite controlled. Scale factors were a multiply of I/Q from full scale (1) downto 0. Usedto adjustboth phase and amplitude values. Inputs were polar and were converted using Xilinx CORDIC. One CORDIC was usedto serially take in 6 channels of input. The new complex channel data fedinto a complex multipliers with single I/Q data stream from IFGEN.
    \end{itemize}
    \item Implemented 10GbE UDP for ZCU111. Used Xilinx 10/25G Ethernet Subsystem (PG210) and shared MAC/IPV4/UDP stack IP.
    \begin{itemize}
      \item Used 10GbE for one ofthe SFP PHY on the ZCU111. External dynamic PDW generator transmitted ethernet packets containing transmit scheduled PDWs. Created message parsing logic on top of the extracted UDP data that contained the fields for the IFGEN scheduler on the ZCU111.
      \item Used seventh DAC to feed back into an ADC usedto statically adjust ADC NCO tuning. Acted as a feedback loop to enable DAC calibration at power-on and any time during run-time. A constant waveform was transmitted and received from DAC to ADC and would be used to compare phase offsets against the other DACS. Manually had to captureoffsets by wiring each DAC into the ADC, but afterinitial capture, these phase adjustments created a calibration table that could be used to adjust output phase of each DAC to reduce potentially perceived dopper offset in downstream multi attenna receiver hardware.
    \end{itemize}
  \end{itemize}

  Oberon 
  \begin{itemize}
    \item Various C drivers. 1-Wire driver for DS25EC20P-T EEPROM. 1-Wire driver for MAX31826MUA+ temperature sensor. GPIO driver for MAX9208EAI+ LVDS Deserializer.
    \item Code/Sim custom packet parser within a Transmit/Receiver front-end scheduler on RFSoC. Transmit packet headers with channel ID's were extracted and used to mux remainder of packet to scheduler on respective DAC channel. Likewise incoming receive data was packetized with incoming scheduled ID and headers. TX/RX data was FIFOd and compared against externallylocked timestamp for scheduling. Hardware propagation delay was adjustable by AXI4L registers.
    \item Design/Code/Sim for a pipelined Multi-Channel scaler. Multiple incoming transmit data streams flowed into the same amplifier hardware downstream, and needed to be scaled by up to 1/3 to prevent saturation on DAC, depending on the number of valid streams during a transmit. Scales were based off additions of 2\textsuperscript{n} shifts of the incoming data.
    \item Optimization of front end Transmit Receive resources. Reduced the buffered descriptor records to only the necessary fields required for that front-end application (BSP level). Transmit and Receive logic were split out to optimize Transmit (receive required all fields to be sent back to off-board Scheduler). Buffers were changed from 1K to 512 depth to infer half BRAMs. Various buffers were optimized to infer LUTRAM instead of BRAM. Ultimately reduced utilization from 192 BRAM downto 24.
  \end{itemize}

  Protean 
  \begin{itemize}
    \item FPGA Design for AMS WB3XR2 Dual RFSoC. Also did high level AXI4L memorymap from Versal to RFPE0 and RFPE1. Versal uses AXI over LAN IP to talk to RFPE1 over 10G, which then talks to RFPE0 over HSS. Each RFPE was allocated 4.25GB of user space addressable by PL (derived from UG1085: system addresses). 
    \item Design/Code of RF Channelizer FPGA boot sequence. Artix is not MPSoC, so no PS. Therefore, supports booting from base image used to initialize Ethernet to an MPSoC that writes down tactical image to AXI4L addressable SPI Flash. Interface to Flash via STARTUPE3 primitive to access FPGA configuration pins. Also uses ICAPE3 to remotely schedule a restart to the FPGA. 
    \item Test and Development of FPGA. Non-tactical image meant to control GPIO and various peripherals attached to FPGA. All of the GPIO is AXI4L controlled, from Xilinx JTAG to AXI4L Crossbar. Documented an I/O test plan and wrote top level FW to implement each test case. Did manual clock routing for external clock source. Routedthrough an HDIO not on a CMT, which can cause Vivado to fail on implementation. Manually constrainted to go through IBUFDS to BUFG to MMCM on the same CMT.
  \end{itemize}

% Vehicle Engineering Intern, Capsule Reusability \hfill January -- July 2015
% \begin{itemize}
%   \item Project development, including hands-on prototyping and designing, conducting and \\
%   presenting experiments to explore changes to Dragon Cargo space capsules.
%   \item Approached several projects simultaneously which demanded intensive problem-solving, \\
%   interpersonal, and time management skills.
%   \item {\bf Projects:} Dragon Capsule Water Sealing, Dragon Capsule Parachute Packing Tool Rework
% \end{itemize}

% \headerwithlabel{RIT Center for Detectors}{Rochester, NY}{March -- May 2016}
% Lab Assistant, Mechanical Engineer
% \begin{itemize}
%   \item Created system-level designs and modeled mechanical components for astronomy research \\ 
%   experiments including a cryogenic sounding rocket payload, a ground-based observatory telescope, \\
%   and small spacecraft.
%   \item Led a team of undergraduate students and served as systems engineer for integration of a \\
%   NASA sounding rocket research payload.
%   \item {\bf Projects:} Cryogenic Star Tracking Attitude Regulation System (CSTARS)
% \end{itemize}

% \headerwithlabel{GE Aviation}{Cincinnati, OH}{January -- May 2014}
% Engineering Co-op, Ultrasonic Non-Destructive Test Lab
% \begin{itemize}
%   \item Operated ultrasonic transducers and 3-axis scanners.
%   \item Analyzed scan imagery for component defects in test samples and flight hardware, including \\
%   composite delaminations and weld voids.
%   \item Developed and optimized test procedures for components with irregular geometry.
%   \item {\bf Projects:} GEnx Flowpath Spacer Inspection Optimization
% \end{itemize}

% \headerwithlabel{RIT Space Exploration (RITSPEX)}{Rochester, NY}{Fall 2014 -- Present}
% Alumni Member
% \begin{itemize}
%   \item Mentor undergraduate students working on space exploration projects.
%   \item Principal Investigator and Project Lead for computer vision and remote sensing payloads.
%   \item {\bf Projects:} SPEX Project Definition Document Template, {\it Where U At Plants?}~(WUAP) HAB Payload, SPEXcast Podcast
% \end{itemize}

% \section{Additional Experience}
% \headerwithlabel{RIT Undergraduate Admissions}{Rochester, NY}{Fall 2013--May 2017}

% \break
% \section{Detailed Project Descriptions}

% \headerwithlabel{Cosmic Dawn Intensity Mapper (CDIM)}{\href{https://github.com/runphilrun/CDIM-design/blob/master/cdim_design.pdf}{github.com/runphilrun/CDIM-design}}{\bf Graduate Paper}
% Contributed to a proposal for a Probe Class (\textasciitilde\$850M) NASA mission for a 1.5 meter space telescope intended to observe near-infrared light from the early universe. Compiled financial, mass, and power budgets for the optics, instruments, cryocooler \& spacecraft. Defined system-level design, generated representative CAD models and figures of the spacecraft.

% \headerwithlabel{1 kW Arcjet Thruster}{\href{https://github.com/RIT-Space-Exploration/msd-P17101/blob/master/p17101.pdf}{github.com/RIT-Space-Exploration/msd-P17101}}{\bf Undergraduate Capstone}
% Developed the concept, system-level design, and nozzle design for a small scale arcjet thruster demonstration. Worked in a multidisciplinary team of mechanical and electrical engineers. Responsible for communication between the team and the customer (RIT Space Exploration). Designed and performed CFD analysis on the thruster nozzle.

% \headerwithlabel{Where U At Plants?~(WUAP) High Altitude Balloon Payload}{\href{https://github.com/RIT-Space-Exploration/hab-cv/}{github.com/RIT-Space-Exploration/hab-cv}}{}
% Where U At Plants? (WUAP) is a high-altitude balloon payload using on-board image processing with a Raspberry Pi 3, Python 3 and OpenCV 3.3 to mask RGB images of the Earth and attempts to mask areas of vegetation using colorspace transformations. WUAP flew as a payload on RIT Space Exploration's HAB4 high altitude balloon mission on April 22, 2018. A \href{https://github.com/RIT-Space-Exploration/hab-cv/blob/master/reports/Project%20Definition%20Document/hab-cv.pdf}{Project Definition Document} and \href{https://github.com/RIT-Space-Exploration/hab-cv/blob/master/reports/HAB4%20Post%20Flight%20Report/report_wuap_postflight-hab4.md}{post-flight report} document the design intent and discuss the results.

% \headerwithlabel{Cryogenic Star Tracking Attitude Regulation System (CSTARS)}{RIT Center for Detectors}{}
% Designed the mechanical model of CSTARS, an experiment endorsed by the New York Space Grant and funded with \$100,000 by NASA's Undergraduate Student Instrument Program. I designed CAD models for the cryogenic thermal regulation system, telescope, and mechanical supports in Solidworks 2015. I was the systems engineer for payload integration with a Black Brant IX at NASA Wallops Flight Facility.

% \headerwithlabel{Machine Learning Hackathon}{Lockheed Martin Space}{}
% Led a team to win a company-wide machine learning hackathon competition. Implemented Expectation Maximization algorithm using K-means in Python3 with sci-kit learn. Presented the project approach and results to a panel of LM Engineering \& Technology upper management. 

% \headerwithlabel{Optical Payload Training Course Project}{Lockheed Martin Space}{}
% Led a multidisciplinary team in a course project to design an optical payload mission concept and instrument. Worked in a multidisciplinary team with subject matter experts in Sunnyvale, CA and Denver, CO. Coordinated team meetings, managed progress and action items. Developed an atmospheric science mission concept and designed the instrument.

% \headerwithlabel{Dragon Capsule Water Sealing}{SpaceX}{}
% Designed and tested retrofits to the Dragon Cargo capsule in order to prevent water ingress on splashdown. Investigated water entry paths, conducted experiments to validate designs, and implemented modifications on flight hardware present on Dragon vehicles since the CRS-7 mission.

% \headerwithlabel{Dragon Capsule Parachute Packing Tool Rework}{SpaceX}{}
% Designed and implemented modifications to the Dragon Cargo parachute packing tool, including working with third party vendors to deliver flight critical components.

% \headerwithlabel{Crew Dragon Weldment Doubler Design}{SpaceX}{}
% Designed and drafted engineering CAD models and drawings for structural components installed on critical load members of a flight vehicle. Design start to installation took 4 days.

% \headerwithlabel{GEnx Flowpath Spacer Inspection Optimization}{GE Aviation}{}
% Optimized parameters for detection of internal wrinkles in composite layups with complex geometry during ultrasonic inspection. Conducted destructive microscopy to validate results and presented findings to principal engineers.

% \headerwithlabel{SPEX Project Definition Document Template}{\small\href{https://github.com/RIT-Space-Exploration/SPEX-Project-Definition-Documents}{github.com/RIT-Space-Exploration/SPEX-Project-Definition-Documents}}{}
% A Project Definition Document (PDD) documents a SPEX project idea and its objectives. This template defines the internal standard of quality for all SPEX PDDs and also serves as a template for other projects to build from. This template was used to define the scope of RIT's payload for the Intercollegiate Rocket Engineering Competition 2018, earning the project a \$1000 grant from Students for the Exploration and Development of Space (SEDS).

% \headerwithlabel{SPEXcast Podcast}{\href{https://blog.spexcast.com/}{blog.spexcast.com}}{}
% I produce, edit, and co-host a space exploration podcast, which is a weekly discussion podcast the science and technology of space exploation. SPEXcast also features interviews with space scientists and industry members, including Tory Bruno, Chris Hadfield, and NASA Scientists.

\end{document}
